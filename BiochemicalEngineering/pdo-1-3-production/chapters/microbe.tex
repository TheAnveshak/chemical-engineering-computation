% chapters/microbe.tex
\chapter{Microbial Host Engineering}
\label{ch:microbe}

\section{Background and reaction pathway}
The canonical biological route to 1,3-propanediol (PDO) proceeds in two enzymatic steps: glycerol is dehydrated to 3-hydroxypropionaldehyde (3-HPA) and then 3-HPA is reduced to PDO. In many native PDO producers (e.g. \textit{Klebsiella}, \textit{Citrobacter}, \textit{Clostridium}) these steps are catalyzed by a B$_{12}$-dependent dehydratase\cite{nakamuraMetabolicEngineeringMicrobial2003} and a linked NAD(P)H-dependent oxidoreductase (DhaT). The patent provides the pathway summary and the stoichiometries (Equations~\ref{eq:dh}--\ref{eq:glyox}) used here\cite{markemptagewilmingtondeus;PROCESSBIOLOGICALPRODUCTION2003}.

\begin{align}
\text{Glycerol} &\xrightarrow{\text{dehydratase}}\; \text{3-HPA} + \mathrm{H_2O} \label{eq:dh} \\[4pt]
\text{3-HPA} + \mathrm{NADH} + \mathrm{H^+} &\xrightarrow{\text{DhaT / ADH}}\; \text{1,3-PDO} + \mathrm{NAD^+} \label{eq:reduct} \\[4pt]
\text{Glycerol} + \mathrm{NAD^+} &\xrightarrow{\text{glycerol dehydrogenase}}\; \text{DHA} + \mathrm{NADH} + \mathrm{H^+} \label{eq:glyox}
\end{align}

Equations \ref{eq:dh}--\ref{eq:glyox} show the key redox link: PDO formation consumes reducing equivalents (NADH) while alternative glycerol oxidation (to DHA) produces NADH. Balancing these fluxes and cofactors is central to designing an efficient glucose$\to$PDO route\cite{markemptagewilmingtondeus;PROCESSBIOLOGICALPRODUCTION2003}.

\section{Key genetic elements (dha regulon and host activities)}
The patent documents the \textbf{dha regulon} (genes typically from \textit{Klebsiella}) used to provide glycerol$\to$3-HPA capability in \textit{E.\,coli} and lists the core genes: \texttt{dhaR, orfY, dhaT, orfX, orfW, dhaB1, dhaB2, dhaB3, orf7}. These encode regulatory, dehydratase subunits, and reactivation factors required for a functional glycerol dehydratase\cite{markemptagewilmingtondeus;PROCESSBIOLOGICALPRODUCTION2003}.

Crucially, the patent shows that \textbf{E.\,coli} encodes an endogenous, non-specific oxidoreductase activity capable of converting 3-HPA to PDO (not the canonical DhaT). The authors identify this native activity biochemically and link it to a specific open reading frame (noted in the patent as \texttt{yohD}/SEQ ID NO:57-58) that encodes a putative oxidoreductase (alcohol dehydrogenase family). Experimental work (gel activity staining, peptide sequencing, gene disruption) supports that this native enzyme provides the required reduction of 3-HPA in dhaT-minus strains. 

\section{Genetic modifications}\cite{markemptagewilmingtondeus;PROCESSBIOLOGICALPRODUCTION2003}
Based on the patent's examples the strain design combines the following deliberate genetic edits and plasmid additions:

\begin{enumerate}
  \item \textbf{Introduce exogenous pathway genes (dha regulon subunits)}: \texttt{dhaB1, dhaB2, dhaB3} (dehydratase subunits) plus \texttt{orf7}, \texttt{orfX} (reactivation factors) and regulators to enable glycerol $\rightarrow$ 3-HPA activity in \textit{E.\,coli}. 
  Purpose: supply the rate-limiting dehydration step absent in wild-type \textit{E.\,coli}\cite{markemptagewilmingtondeus;PROCESSBIOLOGICALPRODUCTION2003,tangMicrobialConversionGlycerol2009,nakamuraMetabolicEngineeringMicrobial2003}.
  native \textit{E.\,coli} lacks a B$_{12}$-dependent glycerol dehydratase complex and reactivation machinery; this activity is essential to convert glycerol $\rightarrow$ 3-HPA\cite{markemptagewilmingtondeus;PROCESSBIOLOGICALPRODUCTION2003}.
  \item \textbf{Removing \texttt{dhaT} (the canonical PDO oxidoreductase)}: experiments in the patent show that processes containing \texttt{dhaT} often accumulate aldehydes (3-HPA and related species) and experience higher cell lysis/ lower viability, likely because of imbalanced redox/reaction coupling and side reactions. Unexpectedly, eliminating \texttt{dhaT} results in lower accumulation of toxic intermediates, improved cell viability, and overall higher PDO titre. \textit{This is a key counter-intuitive design choice that the patent highlights}\cite{markemptagewilmingtondeus;PROCESSBIOLOGICALPRODUCTION2003}.
  \item \textbf{Introduce glycerol synthesis module}: genes encoding glycerol-3-phosphate dehydrogenase (G3PDH) and glycerol-3-phosphatase (G3P phosphatase) to convert glycolytic intermediates (DHAP) into glycerol in the same host,  generates an internal glycerol pool from DHAP, coupling glycolytic flux to PDO formation. This creates a net route glucose $\rightarrow$ glycerol $\rightarrow$ 3-HPA $\rightarrow$ PDO, enabling single-organism conversion\cite{markemptagewilmingtondeus;PROCESSBIOLOGICALPRODUCTION2003}.
  \item \textbf{Targeted knockouts in host competing pathways}: inactivate endogenous genes such as glycerol kinase (\texttt{glpK}), glycerol dehydrogenase (\texttt{gldA}/\texttt{dhaD}) and triosephosphate isomerase (\texttt{tpi}). 
  Purpose: reduce competing consumption of glycerol/DHA or redirect carbon flux into the glycerol production branch and PDO pathway. Examples KLP23 / RJ8 strains in the patent use combinations of such inactivations\cite{markemptagewilmingtondeus;PROCESSBIOLOGICALPRODUCTION2003,nakamuraMetabolicEngineeringMicrobial2003}.
  \item \textbf{Plasmids / expression cassettes}: the patent examples use plasmids (e.g. pKP32, pDT29, pAH48, pKP32) to carry pathway genes or glycerol modules and to tune expression; choice of plasmid alters observed titer and viability in fermentations. 
\end{enumerate}


\section{Final Strain}

The production strain used in the study\cite{bisgaardDatabasedDynamicCompartment2022} would be used for kinetic modeling.
The strain is a genetically modified \textit{E.~coli} K-12 derivative\cite{bisgaardDatabasedDynamicCompartment2022}.

\section{Heterologous genes introduced}
\begin{itemize}
  \item \textbf{DAR1 (G3PDH) and GPP2 (G3P phosphatase)} - genes from \textit{S.~cerevisiae} introduced to convert dihydroxyacetone phosphate (DHAP) into glycerol (glycerol-3-phosphate $\to$ glycerol). This supplies the glycerol precursor in a glucose-based route\cite{wangProduction13propanediolGlycerol2007}. 
  \item \textbf{dhaB1, dhaB2, dhaB3 (glycerol dehydratase) and reactivation factors dhaBX / orfX} - from \textit{Klebsiella pneumoniae}. These perform the dehydration: glycerol $\to$ 3-hydroxypropionaldehyde (3-HPA). The reactivation factors maintain dehydratase activity\cite{bisgaardDatabasedDynamicCompartment2022,nakamuraMetabolicEngineeringMicrobial2003}.
  \item \textbf{yqhD (endogenous oxidoreductase)} - an E.~coli enzyme (or other non-specific alcohol dehydrogenase activity) reduces 3-HPA to 1,3-PDO. In some constructs, deliberate deletion of dhaT (a dedicated 1,3-PDO dehydrogenase) was shown to actually improve yields because the endogenous enzyme gave lower harmful intermediate accumulation. See the patent examples for discussion\cite{nakamuraMetabolicEngineeringMicrobial2003,tangMicrobialConversionGlycerol2009}. 
\end{itemize}

\section{Deletions and host modifications}
\begin{itemize}
  \item \textbf{glpK, gldA deletions:} prevent produced glycerol from re-entering central carbon metabolism (avoid loss of carbon back into growth or by-product pathways)\cite{bisgaardDatabasedDynamicCompartment2022,nakamuraMetabolicEngineeringMicrobial2003}.
  \item \textbf{arcA disruption:} removes oxygen-regulated repression that can trigger unwanted by-product formation under oxygen limitation. This supports high respiration and avoids acetate overflow\cite{bisgaardDatabasedDynamicCompartment2022}.
  \item \textbf{PTS removal + galP + glk introduction:} replacing the phosphotransferase system (PTS) for glucose uptake with a galP (permease) + glk (glucokinase) system improves substrate uptake rate and often increases yield (reduces PEP drain)\cite{bisgaardDatabasedDynamicCompartment2022}.
  \item \textbf{gapA downregulation:} tuning glycolytic flux to favour glycerol formation and redox balance for PDO production\cite{bisgaardDatabasedDynamicCompartment2022}.
\end{itemize}
