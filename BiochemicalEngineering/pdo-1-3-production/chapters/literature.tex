\chapter{Literature Review}

\section{Historical Background}
The interest in biotechnological production of 1,3-PDO began with the discovery that certain microorganisms such as 
\textit{Klebsiella pneumoniae} and \textit{Clostridium butyricum} could ferment glycerol to 1,3-PDO through the action of dehydratase and dehydrogenase enzymes.
However, these natural producers are often pathogenic or require strict anaerobic conditions, which complicate scale-up and industrial implementation \cite{ewingFermentationProductionBiobased2022}\cite{13PropanediolMarket,Biobased13propanediolPDO}.

\section{Industrial Pathways}
\subsection{Glycerol-based Production}
Glycerol, a by-product of biodiesel production, can be converted into 1,3-PDO via two enzymatic steps:
\begin{align}
\ce{Glycerol ->[Dehydratase] 3-Hydroxypropionaldehyde + H2O}
\end{align}
\begin{align}
\ce{3-Hydroxypropionaldehyde + NADH + H+ ->[Dehydrogenase] 1,3-Propanediol + NAD+}
\end{align}
This process has been successfully employed in China by companies such as Zhangjiagang Glory Biomaterial and Shenghong Group with capacities around 20--65~kt\,per\,year \cite{ewingFermentationProductionBiobased2022}.

\subsection{Sugar-based Production using Engineered Strains}
Recent advances in metabolic engineering have enabled the production of PDO directly from glucose.
In engineered \textit{E.~coli}, glucose is first converted to dihydroxyacetone phosphate (DHAP), which is then reduced to glycerol through heterologous genes from \textit{Saccharomyces cerevisiae}. 
Subsequently, introduced genes from \textit{K.~pneumoniae} convert glycerol to 3-hydroxypropionaldehyde, which is then reduced to PDO by an \textit{E.~coli} enzyme \cite{ewingFermentationProductionBiobased2022}.

\section{Comparison of Hosts}
\subsection{Natural Producers}
\textit{Klebsiella pneumoniae}, \textit{Citrobacter freundii}, and \textit{Clostridium butyricum} have been widely studied as natural PDO producers, achieving high yields but facing biosafety limitations.

\subsection{Engineered Hosts}
\textit{E.~coli} has emerged as the preferred production host due to:
\begin{itemize}
    \item Ease of genetic manipulation.
    \item Ability to utilize inexpensive carbohydrate feedstocks.
    \item Availability of well-characterized industrial fermentation systems.
\end{itemize}
