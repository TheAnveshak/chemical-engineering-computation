\chapter{Market Analysis}

\section{Global Market Overview}
The global market for 1,3-propanediol has expanded rapidly since its commercialization by DuPont in the early 2000s.
Recent industry analyses project continued growth, driven by demand for bio-based polymers and environmentally friendly solvents \cite{mansfieldDuPontTateLyle2007}.
At Dupont \& Tate and Lyle Bioproducts' production plant which has been operating since 2006, is reportedly undergoing expansion to increase the capacity from 61--77 kt per annum.\cite{mansfieldDuPontTateLyle2007}

\section{Production Capacities and Key Players}
\begin{itemize}
    \item DuPont's sugar-based fermentation process produces PDO for \textit{Sorona\textsuperscript{\textregistered}} fiber.
    \item Chinese producers such as Zhangjiagang Glory Biomaterial and Shenghong Group operate glycerol-based PDO plants with capacities of 20--65~kt\,per\,year \cite{ewingFermentationProductionBiobased2022}\cite{Biobased13propanediolPDO}\cite{13PropanediolMarket}.
    \item Suzhou Suzhen Biological Engineer Co., Ltd. (a subsidiary of Shenghong Group) officially opend a 20 ktons glycerol based  PDO plant and would increase it to 40 kton in 2nd phase\cite{Biobased13propanediolPDO}.
    \item \textbf{DSM} and \textbf{METEX NØØVISTA} (a subsidiary of METabolic EXplorer) formed an exclusive alliance to produce and market \textit{TILAMAR\textsuperscript{\textregistered} PDO with NØØVISTA\texttrademark}, the first made-in-Europe, 100\% bio-based, non-GMO cosmetic-grade 1,3-propanediol\cite{DSMMETEXNOOVISTA}.
    \item  DuPont Tate \& Lyle Bio Products LLC's Susterra\textsuperscript{\textregistered} 1,3-propanediol is a 100\% bio-based glycol produced via sugar fermentation, offering a sustainable and cost-effective alternative to petrochemical routes\cite{llcPrimientCovationLLC}.
    \item \item \textbf{METabolic Explorer, Genomatica, BASF} — biotechnology companies active in PDO and related diols.
\end{itemize}

\section{Cost Factors}
The main economic drivers for PDO production include:
\begin{enumerate}
    \item \textbf{Feedstock cost:} sugars or glycerol contribute 50--70\% of total production cost.
    \item \textbf{Fermentation yield and titer:} higher concentrations reduce downstream energy demand.
    \item \textbf{Downstream processing:} cell inactivation, ion exchange, and distillation dominate energy and capital costs.
\end{enumerate}

\section{Process Scale and Feasibility}
Plants typically operate between 20 and 77~kt\,per\,year \cite{Biobased13propanediolPDO} capacities. \cite{13PropanediolMarket}
With further optimization of engineered \textit{E.~coli} and reduced feedstock costs, glucose-to-PDO routes are economically viable at industrial scales.

\section{Conclusion}
The global Market is projected to reach USD  690.6 million by 2025 at a cagr  od 100.4\% \cite{13PropanediolMarket}.
The Sale Price or Production cost of 1,3-PDO is 1.83 USD/kg\cite{nimbalkarReviewMicrobial12024}.
The economic outlook for bio-based PDO is promising, provided feedstock prices remain low and process efficiencies improve.
Both glycerol- and sugar-based routes can be commercially attractive depending on regional availability of substrates and infrastructure.
