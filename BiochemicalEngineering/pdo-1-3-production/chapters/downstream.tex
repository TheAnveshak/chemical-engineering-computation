% chapters/separation.tex
\chapter{Downstream Processing and Separation}
\label{ch:separation}

\section{Overview and separation challenges}
Commercial recovery of 1,3-propanediol (PDO) is challenging because fermentation broths typically contain low PDO concentrations (ca.\ 5-15\,wt\%), a complex matrix of cells, salts, organic acids and residual nutrients, and because PDO is relatively hydrophilic, high boiling and low volatility. \\
These properties make recovery energy-intensive and costly: a robust downstream train must (i) remove cells and solids,\\ (ii) exclude inorganic and organic impurities, and \\(iii) perform final concentration and polishing to a commercial product grade.\\
Typical unit operations used in the literature and patents include centrifugation and membrane filtration, ion exchange and electrodialysis, solvent/salting-out extraction, adsorption, pervaporation, and vacuum distillation or preparative chromatography for final polishing \cite{nimbalkarReviewMicrobial12024}.

\section{Primary removal of cell \& Intermediate impurity removal}
The first step is removal of biomass and particulate materials to avoid fouling and excessive solvent load downstream which are separated by membrane filtration, high-speed centrifugation and evaporation.
Choice depends on scale, broth rheology, and downstream sensitivity to particulates.
The broth still contains inorganic salts, organic acid salts, glycerol, small organic molecules and color bodies and are mostly separated by the broth still contains inorganic salts, organic acid salts, glycerol, small organic molecules and color bodies.

\subsection{Ion exchange}
Cation and anion exchange resins remove mineral ions and organic acid salts effectively. 
DuPont reported removal of \textgreater 98\% of mineral salts and organic acid salts using strong-acid cation and weak-base anion exchangers in sequence; however, resins saturate quickly and their regeneration consumes substantial NaOH and HCl\cite{nimbalkarReviewMicrobial12024}.

\subsection{Electrodialysis}
Electrodialysis can selectively remove ionic species (organic acid salts, inorganic ions) and has been used to desalinate fermentation broths. Reports show removal of about 90\% of organic acid salts, but electrodialysis suffers from significant energy consumption, capital cost and can cause PDO losses; it also generates brine waste streams that require disposal or treatment \cite{nimbalkarReviewMicrobial12024}.


\subsection{Liquid-liquid extraction and salting-out}
Because PDO is highly hydrophilic, direct solvent extraction is often inefficient. Salting-out (two-phase extraction) uses inorganic salts (e.g.\ K\textsubscript{2}CO\textsubscript{3}, K\textsubscript{2}HPO\textsubscript{4}) with a water-miscible or partially miscible organic extractant (e.g.\ isopropanol, n-butyl acetate) to partition PDO into an organic or salted phase. Several studies report recoveries $>$90\% and even 98.27\% (e.g.\ K\textsubscript{2}CO\textsubscript{3}/isopropanol systems) depending on salt, pH and solvent choice \cite{nimbalkarReviewMicrobial12024}. Salting-out is attractive for high recovery but leaves large inorganic salt loads in aqueous effluent requiring treatment; solvent selection and recycle are crucial for economics and environmental performance.

\subsection{Reactive Distillation}
Reaction between PDO  \& aldehyde to form dioxolane which is extracted in organic solvent and hydrolsed.

