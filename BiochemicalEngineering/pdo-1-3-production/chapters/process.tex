\chapter{Process design}
\label{ch:pdo_process}


A complete process chapter for producing 1,3-propanediol (
1,3-PDO) from glycerol: seed-train scale-up (10$\times$ steps), fed-batch scale-up into a bubble-column reactor (geometry and aeration based on Bisgaard), fermentation operating strategy (strain design and staging based on Nakamura, Tang), pH/temperature guidance (Dharmadi), and a literature-informed downstream separation train (Przystałowska)\cite{bisgaardDatabasedDynamicCompartment2022,tangMicrobialConversionGlycerol2009,nakamuraMetabolicEngineeringMicrobial2003,dharmadiAnaerobicFermentationGlycerol2006,przystalowskaBiotechnologicalConversionGlycerol2015}.

\section{Design assumptions}
\begin{itemize}
  \item Reactor geometry anchored to Bisgaard: diameter $D = 5.30\ \mathrm{m}$, choose initial liquid height $H = 3D = 3\times 5.30 = 15.9\ \mathrm{m}$ which till the end of process becomes $H = 5D = 5\times 5.30 \approx 26\ \mathrm{m}$\cite{bisgaardDatabasedDynamicCompartment2022}.
  \item Inoculum policy: target inoculum = 10\% (v/v) of reactor initial liquid (Bisgaard practice for industrial seeding which I am extrapolating as discussed in class by Prof. VGG). 10$\times$ volume steps until the final inoculum is prepared; the last step is adjusted to match the 10\% reactor inoculum volume. \cite{bisgaardDatabasedDynamicCompartment2022}.
  \item Strain: engineered \textit{E.\ coli} carrying glycerol $\rightarrow$ 1,3-PDO pathway with dhaB (B12-independent when applicable)\cite{nakamuraMetabolicEngineeringMicrobial2003,tangMicrobialConversionGlycerol2009}.
  \item Feedstock: crude biodiesel glycerol. \cite{przystalowskaBiotechnologicalConversionGlycerol2015}
\end{itemize}

\section{Detailed calculations}
Below are explicit calculation steps (formulas and evaluated numbers). keep these in your design chapter for traceability.

\subsection{Reactor volume from geometry}
Given:
\[
D = 5.30\ \mathrm{m},\qquad H = 3D = 3\times 5.30 = 15.9\ \mathrm{m}.
\]
Cross-sectional area:
\[
A = \pi \left(\frac{D}{2}\right)^2 = \pi\left(\frac{5.30}{2}\right)^2
     \approx 22.067\ \mathrm{m^2}.
\]
Volume:
\[
V_{\text{reactor,init}} = A \times H
    = \pi\left(\frac{D}{2}\right)^2 H
    = \pi (2.65)^2 \times 15.9.
     \approx 350.783\ \mathrm{m^3}.
\] \cite{bisgaardDatabasedDynamicCompartment2022}.

\subsection{Inoculum target (10\% v/v)}
\[
V_{\text{inoc}} = 0.10 \times V_{\text{reactor,init}}
= 0.10 \times 350.783\ \mathrm{m^3}
\approx 35.0783\ \mathrm{m^3}\approx 35078.3\ \mathrm{L}.
\]

\subsection{Seed-train (10$\times$ progression)}
A strict 10$\times$ chain starting from common lab volumes gives:
\[
0.1\ \mathrm{L}\to 1\ \mathrm{L}\to 10\ \mathrm{L}\to 100\ \mathrm{L}\to 1000\ \mathrm{L}\to 10000\ \mathrm{L}\to 35078\ \mathrm{L}\ (\text{overshoot}).
\]
To hit the exact inoculum target and to avoid the overshppting adjust the last inoculum.

\subsection{Biomass in final inoculum}

Bisgaard reports biomass concentration based on optical density at 550 nm, using the conversion
\[
1\ \text{OD}_{550} \equiv 3.0\ \text{g dry weight per L} 
\]\cite{bisgaardDatabasedDynamicCompartment2022}.

\subsection{Aeration and superficial gas velocity}
Bisgaard reports representative mixing-test air flow points:
\[
Q_{\text{air,ref}} = 4.75\ \text{Nm}^3\ \text{s}^{-1} \quad
\]\cite{bisgaardDatabasedDynamicCompartment2022}

Convert to hourly volumetric flow:

\[
Q_{\text{air,ref} (m^3/h)} = 4.75\ \frac{m^3}{s} \times 3600\ \frac{s}{h}
= 4.75 \times 3600 = 17100\ m^3\ h^{-1}.
\]

Cross-sectional area (from above) $A \approx 22.067\ \mathrm{m^2}$. Superficial gas velocity:
\[
U_{\text{sg}} = \frac{Q_{\text{air,ref}}}{A}
= \frac{4.75\ \mathrm{m^3/s}}{22.067\ \mathrm{m^2}}
\approx 0.2153\ \mathrm{m.s^{-1}}.
\] \cite{bisgaardDatabasedDynamicCompartment2022}.

\section{Seed-train schedule}
\begin{table}[h!]
\centering
\begin{tabular}{l l l}
\hline
Step & Volume (L) & Purpose / notes \\
\hline
1 & 0.1 & Lab flask inoculum \\
2 & 1 & intermediate \\
3 & 10 & intermediate \\
4 & 100 & intermediate \\
5 & 1000 & intermediate \\
6 & 10000 & intermediate \\
7 & 35078 & Final inoculum (10\% v/v of reactor) \\
\hline
\end{tabular}
\caption{Seed Train Steps}
\end{table}

\section{Fermentation strategy}
\begin{enumerate}
  \item \textbf{Seed phases:} aerobic cultivation (staged, see volumes above). Typical setpoints: T = 30--37 C, pH near 7.0 (control with \ce{NH4OH}) and DO greater than 30\% saturation during biomass build \cite{tangMicrobialConversionGlycerol2009,bisgaardDatabasedDynamicCompartment2022}.
  \item \textbf{Main reactor fill and growth:} fill the bubble-column to the initial liquid height $H=15.9\ \mathrm{m}$ and inoculate with the prepared inoculum (10\% v/v). Operate an aerobic fed-batch growth phase.\cite{bisgaardDatabasedDynamicCompartment2022}.
  \item \textbf{Transition to production:} two valid choices depending on the strain:
    \begin{itemize}
      \item \emph{Aerobic engineered 1,3-PDO route:} maintain aerobic conditions and feed glycerol (or feed glycerol co-substrate) while operating DO at strain-appropriate setpoint (e.g., DO > 20\%) \cite{nakamuraMetabolicEngineeringMicrobial2003}.
      \item \emph{Tang two-stage glycerol conversion:} switch from aerobic growth to a production stage by (i) replacing/perfusing medium with glycerol feed, (ii) applying the thermal induction (example: 30$^\circ$C $\to$ 42$^\circ$C in Tang) and (iii) adjusting oxygen to microaerobic values if required by the glycerol pathway regulation \cite{tangMicrobialConversionGlycerol2009}.
    \end{itemize}
  \item \textbf{pH control during glycerol conversion:} glycerol fermentation and by-product distribution are pH sensitive; maintain pH in the approximately 6.0--7.0 window \cite{dharmadiAnaerobicFermentationGlycerol2006}.
\end{enumerate}


\section{Feeding strategy}
\begin{itemize}
  \item \textbf{Growth:} glucose/dextrose feed (or glycerol if preferred) under DO-based or biomass-growth feedback.
  \item \textbf{Production:} feed glycerol in a controlled fed-batch profile to avoid glycerol inhibition \cite{dharmadiAnaerobicFermentationGlycerol2006,bisgaardDatabasedDynamicCompartment2022}.
\end{itemize}

\section{Downstream Train}
\begin{itemize}
  \item \textbf{Product/byproducts:} 1,3-PDO (target product) plus byproducts such as acetate, ethanol, succinate and others depending on oxygenation and genetics.\cite{dharmadiAnaerobicFermentationGlycerol2006,przystalowskaBiotechnologicalConversionGlycerol2015,murarkaFermentativeUtilizationGlycerol2008}.
  \item \textbf{Downstream (recommended staged train):}
    \begin{enumerate}
      \item Cell removal: membrane filtration or High Speed Centrifuge.
      \item Volatile removal: low-temperature distillation/vacuum stripping to remove ethanol and light volatiles.
      \item Acid/byproduct removal: pH adjustment and liquid-liquid extraction or ion exchange to remove succinate/acetate (reviewed options in Przystałowska) \cite{przystalowskaBiotechnologicalConversionGlycerol2015}.
      \item PDO concentration/polishing:  evaporation and  distillation to reach commercial purity; final polishing ion exchange.
    \end{enumerate}
\end{itemize}

