% chapter_reactor.tex
\chapter{Reactor \& Kinetics}
\label{chap:reactor}


\section{Bioreactor and process description}
I model the same system studied in the paper by Bisgaard et\,al. (2022): a bubble column of diameter \(D=5.3\)\,m and nominal total volume \(V\approx600\)\,m$^3$ operated as a fed-batch producing 1,3-propanediol with a recombinant \textit{E.\ coli} strain. The process ran for 32\,hr, the gassed liquid height increased during the feed, and air was supplied through bottom spargers with a time-varying gas flow profile. The paper measured DO and pressure at a probe 5.85\,m from the bottom and used flow-following sensors to generate axial velocity fields\cite{bisgaardDatabasedDynamicCompartment2022}.

\begin{figure}[htbp]  % h=here, t=top, b=bottom, p=page; "!" forces placement
    \centering
    \includegraphics[width=0.8\textwidth]{figures/bubcol.png}
    \caption{Illustration of Bubble Column Bioreactor}
    \label{fig:bubble-column}
\end{figure}image

\section{Hydrodynamics and transformations}
\subsection{Axial position from pressure (Pascal's law)}
From the sensor pressure \(P(t)\) we can calculate the axial position \(z(t)\) using
\[
z(t) \;=\; \frac{P_{\max}-P(t)}{\rho_f(t)\, g},
\tag{1}
\]
where \(P_{\max}\) is the maximum measured pressure (bottom of the vessel), \(\rho_f(t)\) is the instantaneous fluid density and \(g\) is gravity. The paper describes compensation of headspace and volume changes before this transform.\cite{bisgaardDatabasedDynamicCompartment2022}.

\subsection{Gas hold-up correlation}
The study used the an't Riet \& Tramper (1991) style correlation for a heterogeneous regime as reported in the paper:
\[
\varepsilon \;=\; v_s^{0.25} \;+\; 0.45\,(g\,v_s\,D)^{1/3},
\tag{2}
\]
where \(\varepsilon\) is the gas hold-up (gas fraction), \(v_s\) is the superficial gas velocity (m\,s$^{-1}$), \(g\) is gravity and \(D\) the column diameter.

\subsection{Fluid density of the gas-liquid dispersion}
Given the gas hold-up \(\varepsilon\)the dispersion (bulk) density as the liquid-gas volumetric average:
\[
\rho_f \;=\; (1-\varepsilon)\,\rho_l \;+\; \varepsilon\,\rho_g.
\tag{3}
\]



\subsection{$k_L$ a model}
Bisgaard et\,al. adopted a linear relation (Guske \& Miller, 2009) and used the simple relation:
\[
k_L a(j,k) \;=\; 0.288 \, v_s(j,k),
\tag{5}
\]

\section{Compartmental model}
\subsection{Model layout}
Partition the column axially into \(K\) axial compartments (index \(k=1,\dots,K\)), where each compartment is assumed ideally mixed. Between compartments there are bidirectional axial flows \(Q_{k\leftrightarrow k+1}(t)\) derived from the Lagrangian sensor velocities; volumes \(V_k(t)\) change with the liquid height and are updated at discrete update steps \(j=1,\dots,J\).


\subsection{governing differential equations}
The paper presents the following standard per-volume ODEs for a single ideal compartment (their Equations 3-6).:

\paragraph{Biomass:}
\[
\frac{dC_{x,k}}{dt} \;=\; \mu_k\, C_{x,k} \;.
\tag{6}
\]

\paragraph{Substrate:}
\[
\frac{dC_{s,k}}{dt} \;=\; -r_{s,k}\, C_{x,k} \;+\; F_{s,j,k=K}.
\tag{7}
\]
Feed \(F_s\) is added to the top compartment \(k=K\) according to the profile and is updated at each compartment update step
\begin{figure}[htbp]  % h=here, t=top, b=bottom, p=page; "!" forces placement
    \centering
    \includegraphics[width=0.8\textwidth]{figures/BiovsSubs.png}
    \caption{Biomass \& Substrate Concentration wrt Time}
    \label{fig:subs-biomass}
\end{figure}image
\paragraph{Product (PDO):}
\[
\frac{dC_{p,k}}{dt} \;=\; r_{p,k}\, C_{x,k} \;
\tag{8}
\]
\begin{figure}[htbp]  % h=here, t=top, b=bottom, p=page; "!" forces placement
    \centering
    \includegraphics[width=0.8\textwidth]{figures/prodvstime.png}
    \caption{Concentration of Product v/s Time}
    \label{fig:product}
\end{figure}image
\paragraph{Dissolved oxygen (DO):}
\[
\frac{dC_{o,k}}{dt} \;=\; k_L a(j,k)\,\big(C_{o}^{*}(j) - C_{o,k}\big) \;-\; r_{o,k}\, C_{x,k}
\tag{9}
\]
where \(C_o^{*}(j)\) is the local saturation concentration in the compartments for the update step \(j\).
\begin{figure}[htbp]  % h=here, t=top, b=bottom, p=page; "!" forces placement
    \centering
    \includegraphics[width=0.8\textwidth]{figures/DOvstime.png}
    \caption{Concentration of Dissolved Oxygen v/s Time}
    \label{fig:dissolved-oxygen}
\end{figure}image
\begin{lstlisting}[caption={Simple Monod kinetics simulation in Python}]
# compute derivatives and instantaneous specific rates (finite differences)

dt = np.gradient(t)  # time step array (h)
dCxdt = np.gradient(Cx, t)
dCsdt = np.gradient(Cs, t)
dCpdt = np.gradient(Cp, t)
dCodt = np.gradient(Co, t)

# avoid division by zero
eps = 1e-12
rp_inst = dCpdt / (Cx + eps)        # specific product formation rate (1/h)
rs_inst = -dCsdt / (Cx + eps)       # specific substrate uptake (mass substrate / (mass biomass * h))
mu_inst = dCxdt / (Cx + eps)        # instantaneous specific growth rate (1/h)
\end{lstlisting}

\section{Kinetic  rates}
\subsection{Growth rate (Monod with product inhibition)}
Bisgaard et\,al. used a Monod form with oxygen limitation and product inhibition:
\[
\mu_k \;=\; \mu_{\max}\,\frac{C_{s,k}}{C_{s,k}+K_s}\,\frac{C_{o,k}}{C_{o,k}+K_o}\,\Big(1 - \frac{C_{p,k}}{K_p}\Big),
\tag{10}
\]
where \(K_s\) is the half-saturation for substrate, \(K_o\) for oxygen, and \(K_p\) the product inhibition concentration.

\begin{lstlisting}[caption={Simple Monod kinetics simulation in Python}]
# fit mu params: mu_max, Ks, Ko, Kp using dX/dt = mu(Cs,Co,Cp) * X

def mu_model(cs, co, cp, params):
    mu_max, Ks, Ko, Kp = params
    term_s = cs / (cs + Ks + 1e-12)
    term_o = co / (co + Ko + 1e-12)
    term_p = np.maximum(0.0, (1 - cp / (Kp + 1e-12)))
    return mu_max * term_s * term_o * term_p

def simulate_Cx(t, mu_max, Ks, Ko, Kp):
    C0 = Cx[0]
    params = [mu_max, Ks, Ko, Kp]

    def rhs(C, tt):
        cs = np.interp(tt, t, Cs)
        co = np.interp(tt, t, Co)
        cp = np.interp(tt, t, Cp)
        mu = mu_model(cs, co, cp, params)
        return mu * C

    Csim = odeint(lambda C, tt: rhs(C, tt), C0, t).flatten()
    return Csim

# initial guesses and bounds
p0 = [4, 5, 1, 20.0]
bounds = ([1e-6, 1e-6, 1e-6, 1e-6], [5.0, 1e3, 1.0, 1e5])

# fit model to experimental biomass data
popt, pcov = curve_fit(simulate_Cx, t, Cx, p0=p0, bounds=bounds, maxfev=2000)
\end{lstlisting}

\subsection{Product formation}
Specific product formation is given by:
\[
r_{p,k} \;=\; Y_{pX}\,\mu_k \;+\; r_{pX},
\tag{11}
\]


\subsection{Substrate and oxygen uptake rates}
As in the paper:
\[
r_{s,k} \;=\; \frac{\mu_k}{Y_{xs}} \;+\; \frac{r_{p,k}}{Y_{ps}} \;+\; r_{m,s},
\tag{13}
\]
\[
r_{o,k} \;=\; \frac{r_{s,k}}{Y_{so}} \;+\; r_{m,o},
\tag{14}
\]
where the \(Y\)'s are stoichiometric yields and \(r_{m,\cdot}\) are maintenance terms. 

\begin{lstlisting}[caption={Simple Monod kinetics simulation in Python}]
dX_total = Cx[-1] - Cx[0]
dP_total = Cp[-1] - Cp[0]
dS_total = Cs[0] - Cs[-1]   # substrate consumed (positive)

Yxs_global = dX_total / dS_total if abs(dS_total) > 1e-12 else np.nan
Yps_global = dP_total / dS_total if abs(dS_total) > 1e-12 else np.nan
Ypx_global = dP_total / dX_total if abs(dX_total) > 1e-12 else np.nan
\end{lstlisting}


\subsection{Fitted parameters}
I fitted the kinetic and transfer parameters to the experimental data.
Table~\ref{tab:fitparams} lists the fitted parameter values using \texttt{curve\_fit} from \texttt{scipy.optimize}, though Bisgaard et al.\ (2022) uses \texttt{LSODA} from \texttt{scipy.integrate} to integrate and then applies \texttt{Nelder-Mead} to minimize SSE and iteratively solve for the parameters.
\begin{figure}[htbp]  % h=here, t=top, b=bottom, p=page; "!" forces placement
    \centering
    \includegraphics[width=0.8\textwidth]{figures/fit.png}
    \caption{Experimental v/s fitted Biomass COncentration}
    \label{fig:bcurve_fit}
\end{figure}image

\begin{table}[ht]
  \centering
  \caption{Fitted parameters and goodness-of-fit metrics.}
  \label{tab:fitparams}
  \begin{tabular}{l c l}
    \toprule
    Parameter & Fitted value & Units \\
    \midrule
    $\mu_{\max}$ & \texttt{$3.97$} & hr$^{-1}$ \\
    $K_s$        & \texttt{$4.89$} & g L$^{-1}$ \\
    $K_o$        & \texttt{$0.99$} & mg L$^{-1}$ \\
    $K_p$        & \texttt{$1227.3\times 10^{2}$} & g L$^{-1}$ \\
    $Y_{p/X}$     & \texttt{$15.51$} & g product g$^{-1}$ biomass \\
    \midrule
    $R^2$        & \texttt{$-6.9$} &  \\
    SSE          & \texttt{$5160$} &  \\
    \bottomrule
  \end{tabular}
\end{table}

\[
\text{SSE} \;=\; \sum_{i=1}^n (y_i - \hat{y}_i)^2, \qquad
R^2 \;=\; 1 - \frac{\text{SSE}}{\sum_{i=1}^n (y_i-\bar{y})^2}.
\]

\begin{lstlisting}[caption={Simple Monod kinetics simulation in Python}]
y = np.asarray(Cx)
yhat = np.asarray(Cx_sim)

n = len(y)
p = len(mu_params) if 'mu_params' in globals() else 1

sse = np.sum((y - yhat)**2)
mse = sse / n
rmse = np.sqrt(mse)
ss_tot = np.sum((y - np.mean(y))**2)
r2 = 1 - sse / ss_tot if ss_tot > 0 else np.nan
adj_r2 = 1 - (1 - r2) * (n - 1) / (n - p - 1) if (n - p - 1) > 0 else np.nan
\end{lstlisting}
